\documentclass[12pt,letterpaper]{article}
\usepackage[utf8]{inputenc}
\usepackage[english]{babel}
\usepackage{amsmath}
\usepackage{amsfonts}
\usepackage[letterpaper,margin=0.8in]{geometry}
\usepackage{fancyhdr}
\usepackage{helvet}

\pagestyle{empty}
\pagestyle{fancy}
\fancyhf{}
\lhead{Name:}
\rhead{\textit{December 10, 2020}}
\cfoot{Page \thepage}

\begin{document}

\section*{How to solve equations}
\begin{large}

Solving an equation or "finding the roots" means finding the values for the unknown variable.
The trick of figuring out what to do first is to gather place each side of the equation
in simplest form, and then move all the terms containing x to one side, and others to 
the other side of the equation.

\begin{enumerate}
	\item Simplify each side of the equation.
	\item Move variables to the left side of the equation - LHS means "left-hand side".
	\item Move numbers to the right side of the equation - RHS means "right-hand side".
	\item Get the variable by itself.
\end{enumerate}

\noindent Remember what you do to one side of the equation you must do to the other side.
\end{large}

\pagebreak

\section*{Solve the following equations}
\begin{large}
\begin{enumerate}
\item \quad $3x + 4 = 2(2x + 7)$

\addvspace{0.6in}

\item \quad $3(5x + 6) = 3x - 2$

\addvspace{0.6in}

\item \quad $5(x + 3) - 4(2x - 9) = 0$

\addvspace{0.6in}

\item \quad $9x - 2(x + 8) = 5x - 11$

\addvspace{0.6in}

\item \quad $\dfrac{2x + 9}{5} = 5$

\addvspace{0.6in}

\item \quad $\dfrac{3x - 11}{7} - 2 = 0$

\addvspace{0.6in}

\item \quad $\dfrac{7x - 2}{2} = \dfrac{5x - 3}{3}$

\addvspace{0.6in}

\item \quad $\dfrac{2(1 - 4x)}{5} - 9 = 3(2 - x)$

\addvspace{0.6in}

\item \quad $\dfrac{2x + 3}{3} = \dfrac{3x - 15}{11}$

\addvspace{0.6in}

\end{enumerate}
\end{large}

\end{document}