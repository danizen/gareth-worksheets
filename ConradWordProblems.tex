\documentclass[12pt,letterpaper]{article}
\usepackage[utf8]{inputenc}
\usepackage[english]{babel}
\usepackage{amsmath}
\usepackage{amsfonts}
\usepackage[letterpaper,margin=0.8in]{geometry}
\usepackage{fancyhdr}
\usepackage{helvet}

\pagestyle{empty}
\pagestyle{fancy}
\fancyhf{}
\lhead{Name:}
\rhead{\textit{January 8, 2020}}
\cfoot{Page \thepage}

\begin{document}

\section*{Word Problems}

\begin{large}
\begin{enumerate}
\item A zoo has 96 chickens and ducks altogether. The ratio of the number of chickens to the number of ducks is 5:3. How many chickens does the zoo have?

\begin{enumerate}
	\item Draw a picture relating the number of chickens and ducks and their total?

	\addvspace{0.3in}

	\item How many total units (units of chickens + units of ducks are there)?

	\addvspace{0.3in}

	\item Since there are 96 total units, what is the size of a unit?

	\addvspace{0.3in}

	\item How many chickens does the zoo have?

	\addvspace{0.3in}

\end{enumerate}

\item Charlie divides a basket of cherries into two portions. The ratio of the mass of the bigger portion to the mass of the smaller portion is 5 : 2. If the mass of the bigger portion is 15 kg, find the mass of the smaller portion.

\addvspace{0.6in}

\item A board was cut into two pieces whose lengths are in the ratio 2:5. The longest piece was 85 inches. How long was the shortest piece?

\addvspace{0.6in}

\item At a track meet, 5/8 of the students were boys. There were 48 more boys. How many students were there altogether?

\addvspace{0.6in}

\item Theresa read two and one-fourth times as many pages on Saturday as on Sunday. She read 120 pages more on Saturday. How many pages did she read altogether?

\addvspace{0.6in}

\end{enumerate}
\end{large}

\end{document}